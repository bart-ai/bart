\documentclass[a4paper]{article}

\usepackage[T1]{fontenc}
\usepackage[spanish]{babel}

\usepackage{graphicx, geometry, array, booktabs, float, hyperref, setspace}
\usepackage[dvipsnames]{xcolor}

\hypersetup{
    colorlinks=true,
    linkcolor=black,
    urlcolor=RoyalBlue,
}


\begin{document}

\begin{titlepage}
    \newgeometry{top=0.6in,bottom=0.6in,right=1in,left=1in}

    \centering
    \hfill
    \begin{minipage}{0.7\textwidth}
            \centering
            \LARGE
            \textsc{\textbf{Facultad de Ingeniería}}\\[0.1cm]
            \textsc{\textbf{Universidad de Buenos Aires}}
        \end{minipage}%
        \begin{minipage}{2.6cm}
            \centering
            \includegraphics[width=2.6cm]{./img/fiuba.png}
        \end{minipage}

    \vspace{3cm}
    \huge \bfseries Trabajo Profesional \\
    \LARGE \bfseries Licenciatura en Análisis de Sistemas, Ingeniería en Informática
    \vspace{2cm}

    \rule{\linewidth}{0.3mm} \\[0.1cm]
    \huge \bfseries Sistema de detección en tiempo real de publicidad en la vía pública \\
    \rule{\linewidth}{0.3mm}\\[0.7cm]

    \large \emph{Tutor:} Ing. Martín Buchwald\\[0.6cm]
    \begin{minipage}{0.4\textwidth}
        \begin{flushleft}
            \centering
            \large del Mazo, Federico \\
            100029
        \end{flushleft}
    \end{minipage}
    \begin{minipage}{0.4\textwidth}
        \begin{flushright}
            \centering
            \large Pastine, Casimiro \\
            100017
        \end{flushright}
    \end{minipage}

\end{titlepage}

\restoregeometry
\tableofcontents
\newpage

\section{Motivación}

Dentro del marco de trabajo profesional de las carreras de Licenciatura en Análisis de Sistemas y de Ingeniería en Informática de la Facultad de Ingeniería, Universidad de Buenos Aires, se presenta el proyecto de un sistema de detección de publicidades en la vía pública en tiempo real.

\setcounter{secnumdepth}{0}
\doublespacing

\section{Referencias}

\textbf{[1]} OpenCV: \url{https://opencv.org/}

\section{Bibliografía}

Rosebrock, Adrian. (2017). \textit{Deep learning for Computer Vision with Python.}

Ansari, Shamshad. (2020). \textit{Building Computer Vision Applications Using Artificial Neural Networks}, Apress.

Moroney, Laurence. (2020). \textit{AI and Machine Learning for Coders}, O'Reilly.

\end{document}
